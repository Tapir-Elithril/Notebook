\documentclass[11pt]{article}
\usepackage{amsmath,amssymb,enumitem,algorithm,algpseudocode}
\usepackage[UTF8]{ctex}
\usepackage[braket]{qcircuit}
\parindent=22pt
\parskip=3pt
\oddsidemargin 18pt \evensidemargin 0pt
\leftmargin 1.5in
\marginparwidth 1in \marginparsep 0pt \headsep 0pt \topskip 20pt
\textheight 225mm \textwidth 148mm
\renewcommand{\baselinestretch}{1.15}
\begin{document}
\title{{\bf 理论作业一 \quad 量子比特与量子门}}
\author{姓名 \quad 学号}
\date{\today}
\maketitle

\begin{tabular*}{13cm}{r}
\hline
\end{tabular*}

\vskip 0.3 in

{\bf 1.} 已知双量子比特系统的量子态如下 $|\psi\rangle = \begin{bmatrix} \frac{1}{2} & x & 3x & \frac{i}{2\sqrt{2}} \end{bmatrix} ^ \intercal \in \mathbb{C}^4$ ,求该系统处于 $|01\rangle$ 态的概率。

\vskip 0.3 in

{\bf 2.} 已知单量子比特的态矢量为 $|\psi\rangle = \begin{bmatrix} 3/5 \\ 4/5 \end{bmatrix}$ ,求该量子比特的Bloch球坐标。

\vskip 0.3 in

{\bf 3.} Bell 态指双量子比特系统的四个特殊量子态,他们是双量子比特系统中纠缠度最高的量子态,因此也称为最大纠缠态,在量子隐形传态、量子算法中有着广泛的应用。一般而言,Bell 态定义如下:
\begin{align}
    |\beta_{xy}\rangle = \frac{|0y\rangle + (-1)^x|1\bar{y}\rangle}{\sqrt{2}}, \quad x,y \in \{0,1\}, \quad \bar{y} = 1-y
\end{align}
\begin{enumerate}[label=\alph*.]
\item 证明 Bell 态是纠缠态。
\item 用 H、X、Z 和 CNOT 门设计四个量子电路,使得初态为 $|00\rangle$ 的双量子比特系统经这些量子电路作用后分别演化为四个 Bell 态。
\end{enumerate}

\vskip 0.3 in

{\bf 4.} 证明下图中的两个量子电路等价。(提示:计算两个量子电路对应的酉矩阵)

\[ \Qcircuit @C=1.0em @R=2.0em {
\lstick{q_0} & \targ & \qw \\
\lstick{q_1} & \ctrl{-1} & \qw
} \]

\[ \Qcircuit @C=1.0em @R=1.5em {
\lstick{q_0} & \gate{H} & \ctrl{1} & \gate{H} & \qw \\
\lstick{q_1} & \gate{H} & \targ & \gate{H} & \qw 
} \]

\vskip 0.3 in

{\bf 5.} 证明厄米算符 $A$ 的任一本征值均为实数,且不同本征值对应的本征态正交。

\vskip 0.3 in

{\bf 6.} Deutsch 算法展示了量子计算机强大的并行计算能力。Deutsch-Jozsa 算法是其推广形式,将可分类的函数推广至多比特情形。

已知函数 $f:\{0,1\}^n \rightarrow \{0,1\}$,该函数是常数函数(对所有输入均输出 $0$ ,或对所有输入均输出 $1$)或平衡函数(对恰好一半的输入输出 $0$ ,对另一半输入输出 $1$)。Deutsch-Jozsa 算法只需对实现函数 $f$ 的结构进行一次查询,即可判断 $f$ 是常数函数还是平衡函数。

下图是实现 Deutsch-Jozsa 算法的量子线路。其中,$U_f:|x,y\rangle \to |x,y\oplus f(x)\rangle$ 是实现函数 $f$ 的 $n+1$ 比特的量子门。

\[ \Qcircuit @C=1.0em @R=.7em {
\lstick{\ket{0}} & {/^n} \qw & \gate{H^{\otimes n}} & \multigate{1} {U_f} & \gate{H^{\otimes n}} & \meter & \qw \\
\lstick{\ket{1}} & \qw & \gate{H} & \ghost{U_f} & \qw & \qw & \qw \\
} \]

推导该量子电路中量子态的演化过程,并说明如何基于测量结果判断 $f$ 是常数函数还是平衡函数。(提示:计算 $f$ 为常数函数或平衡函数时的测量结果)

\end{document}
