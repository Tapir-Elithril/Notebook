\documentclass[11pt]{article}
\usepackage{amsmath,amssymb,enumitem,algorithm,algpseudocode}
\usepackage[UTF8]{ctex}
\usepackage[braket]{qcircuit}
\parindent=22pt
\parskip=3pt
\oddsidemargin 18pt \evensidemargin 0pt
\leftmargin 1.5in
\marginparwidth 1in \marginparsep 0pt \headsep 0pt \topskip 20pt
\textheight 225mm \textwidth 148mm
\renewcommand{\baselinestretch}{1.15}
\begin{document}
\title{{\bf 理论作业二 \quad 量子测量与量子算法}}
\author{姓名 \quad 学号}
\date{\today}
\maketitle

\begin{tabular*}{13cm}{r}
\hline
\end{tabular*}

\vskip 0.3 in

{\bf 1.} 假设有初始化为 $|1\rangle$ 态的量子寄存器若干,给出分别使用酉算子 $H$、$X$、$T$、$S$ 进行测量的结果。

\vskip 0.3 in

{\bf 2.} 证明 Grover 算法中的算子 $G$ 每次作用时使量子态向 $|\beta\rangle$ 方向旋转角度 $\theta$。

\vskip 0.3 in

{\bf 3.} 根据 Grover 算法中 $M$、$N$ 的定义,令 $\gamma = M/N$,证明在 $|\alpha\rangle$、$|\beta\rangle$ 基下,Grover 算法中的算子 $G$ 可以写为 $\begin{bmatrix}
    1-2\gamma & -2\sqrt{\gamma-\gamma^2} \\ 2\sqrt{\gamma-\gamma^2} & 1-2\gamma
\end{bmatrix}$。

\vskip 0.3 in

{\bf Bonus:} 给出 RSA 算法加密、解密过程的证明,即证明明文为 $a \equiv C^d \mod n$。

\end{document}
